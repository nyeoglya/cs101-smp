% TeX root = ../Main.tex

% First argument to \section is the title that will go in the table of contents. Second argument is the title that will be printed on the page.
\section[2. Introduction - {\it 들어가며}]{2. Introduction}

\subsection{Computational Thinking}

문제를 해결하고, 작업을 수행하고, 지식을 표현하며, 컴퓨터를 활용하여 가능한 한 효율적으로 작업하는 능력

\begin{itemize}
    \item \textbf{추상화}\textit{\textsuperscript{Abstraction}}: 복잡한 문제, 개념 단순화. 핵심 개념, 특성에 집중
    \item \textbf{알고리즘}\textit{\textsuperscript{Algorithm}}: 문제 해결 위한 명확하고 구체적인 방법과 절차
    \item \textbf{자동화}\textit{\textsuperscript{Automation}}: 일련의 작업 자동적으로 수행
\end{itemize}
사람이 직접 문제를 풀려면 시간이 너무 오래 걸리는 일을 컴퓨터에게 시키기 위해서는 문제를 컴퓨터가 풀 수 있는 방식으로 재정의해야 한다. 즉, 추상화된 데이터를 주고 컴퓨터가 문제를 풀어내게 하여 주어진 일을 자동화하는 일련의 과정에 대한 사고 체계가 바로 computational thinking이다.

\subsection{Programming Language \& OS}

\begin{itemize}
    \item \textbf{OS}\textsuperscript{Operating System}: 컴퓨터 하드웨어와 소프트웨어와 같은 컴퓨터 자원을 효율적으로 관리하기 위한 시스템.
    \item \textbf{DOS}\textsuperscript{Disk OS}: 텍스트 기반의 운영체제이다. 원시적인 형태의 문자 인터페이스(CLI)가 존재한다.
    \item \textbf{UNIX, Linux}: Unix는 독점 운영 체제인 반면 Linux는 오픈 소스이다. 둘은 굉장히 유사하며, 둘 모두 다양한 시스템 사이에 이식이 가능하고 멀티 테스킹과 다중 사용자 지원하는 OS이다.
\end{itemize}

\subsection{Python 특징}

Python이라고 하는 언어는 코딩하기에 유용한 여러 장점을 가지고 있다. 이 과목에서 프로그래밍을 가르치는데 Python을 선정한 것에는 다음과 같은 이유가 있다.

\begin{enumerate}
\def\labelenumi{\arabic{enumi}.}
    \item \textbf{동적 타이핑}\textit{\textsuperscript{dynamic typing}}: 변수의 타입이 자동으로 결정된다. 이는 개발자에게 유연성을 제공한다.
    \item \textbf{OOP}\textit{\textsuperscript{Object Oriented Programming}}: class, object를 사용한다. 데이터 처리하는 method를 캡슐화하여 코드 재사용성 \& 유지 보수성이 향상된다.
    \item \textbf{코딩 용이}: 쉽게 사용가능한 여러 라이브러리가 존재한다.
    \item \textbf{가독성, 간결성}: 문법이 쉽고 들여쓰기로 가독성을 높였다.
    \item \textbf{크로스 플랫폼 지원}: 다양한 운영체제에서 동일한 코드 사용할 수 있다.
    \item \textbf{확장성}: 다른 언어에서 작성된 모듈 사용하거나, 파이썬에서 작성된 모듈을 다른 언어에서 사용 가능하다. 여기에는 적절한 미들웨어(운영 체제와 응용 소프트웨어의 사이를 조정하는 소프트웨어)가 필요하다.
\end{enumerate}

\subsection{Program Structure}

일반적으로 프로그램이 문제를 해결하는 과정은 다음의 순서를 따른다.

\begin{enumerate}
\def\labelenumi{\arabic{enumi}.}
    \item \textbf{데이터 입력}: 해결하고자 하는 문제와 관련된 데이터를 모두 입력받는다.
    \item \textbf{문제해결 과정}: 입력받은 데이터를 출력한다. 이 과정에서 데이터를 처리하는 일련의 과정을 담은 \textbf{알고리즘}이 사용된다.
    \item \textbf{데이터 출력}: 가공된 데이터를 출력한다. 파일에 값을 저장하거나, 화면에 출력하는 작업 등이 바로 이 과정이 된다.
\end{enumerate}

\subsection{Programming Paradigm}

\textbf{패러다임}\textit{\textsuperscript{paradigm}}이란, 문제를 푸는 하나의 방법을 의미한다. 같은 문제가 주어졌을 때, 내가 어떤 시선을 가지고 있는지에 따라 문제를 해결하는 방법에 변화가 생기게 된다. 프로그래밍 또한 이전에 사용되던 문제 풀이 방법의 한계가 발견되면서 점점 발전되어왔다. 아래는 각 패러다임의 이름과 그 설명을 나타내 것이다.

\begin{itemize}
    \item \textbf{절차지향}\textit{\textsuperscript{procedual}}: 처리해야 할 문제의 해결 과정을 큰 문제를 독립적인 기능별로 나눠서 일련의 순서에 따라서 처리. 함수가 필수적.
    \item \textbf{객체지향}\textit{\textsuperscript{object oriented}}: 관계 있는 데이터와 함수를 하나로 묶어서 선언하는 클래스 도입. 클래스는 객체를 생성하는 데이터 타입의 역할을 한다. 객체지향 개념(상속, 다형성 등) 활용하여 효율적인 코드 작성.
    \item \textbf{함수지향}\textit{\textsuperscript{functional}}: 자료 처리를 수학적 함수의 계산으로 취급. 람다 함수(고차원 함수)가 지원. 코드 간결하게 만들어줌. 복잡한 로직에는 일반적인 함수 사용.
\end{itemize}
