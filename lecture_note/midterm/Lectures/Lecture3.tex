\section[3. Variables and Types - {\it 변수와 타입}]{3. Variables and Types}

\subsection{Variables}

\textbf{변수}\textit{\textsuperscript{variable}}란 데이터를 저장할 수 있는 메모리 공간에 붙여진 이름이다. 변수는 \textbf{식별자}\textit{\textsuperscript{identifier}}로 구분하여 사용된다. 식별자는 곧 변수의 고유한 이름이며 아래와 같은 이름 규칙의 제한을 받는다.

\begin{itemize}
  \item \textbf{이름 규칙}: 문자, 숫자, 밑줄("\_")만 사용 가능. 예약어를 쓸 수 없으며 대소문자를 구별한다. 또한, 변수의 이름은 문자로 시작해야 한다.
\end{itemize}
다른 언어들과 다르게, Python은 변수와 객체의 메모리를 자동으로 관리하기 때문에 프로그래머가 메모리를 덜 신경써도 된다.

\subsection{Data Types}
Python에서 제공하는 기본 데이터 타입은 다음의 7가지 종류가 있다.

\begin{enumerate}
\def\labelenumi{\arabic{enumi}.}
  \item \textbf{number}: 숫자를 저장하며, int(정수), float(실수)가 해당한다.
  \item \textbf{bool}: 참, 거짓을 저장한다.
  \item \textbf{string}: 문자열(문자들의 나열)을 저장한다.
  \item \textbf{list}: 여러 개의 요소를 순서대로 저장한다.
  \item \textbf{tuple}: list와 유사하지만 값을 변경하는 것이 불가능하다.
  \item \textbf{dictionary}: key-value 쌍으로 구성된 데이터를 저장한다.
  \item \textbf{set}: list와 유사하지만 중복되지 않는 요소로 구성된다.
\end{enumerate}
변수는 객체를 가리키는 아이디를 담고 있는 저장 공간이다. 다음과 같이 입력하면 변수의 아이디를 출력할 수 있다.

\begin{tcolorbox}[colframe=black, colback=white]
\begin{minted}{python}
x=10
id(x)
\end{minted}
\end{tcolorbox}

\subsection{Mutability}
Python의 타입은 \textbf{변경 가능한 값}\textit{\textsuperscript{mutable value}}과 \textbf{불가능한 값}\textit{\textsuperscript{immutable value}} 이렇게 2가지 종류로 구분지을 수 있다. 모든 값은 객체로 처리된다. 이는 나중에 더 자세히 다룰 것이다.

\begin{itemize}
    \item \textbf{mutable}: list, dictionary, set이 속한다.
    \item \textbf{immutable}: number, bool, string, tuple이 속한다.
\end{itemize}
만약, 변경이 불가능한 값이 할당되어 있는 변수에 새로운 값을 집어넣으려 하면 기존의 값은 지워지고 새 값이 들어가게 된다.

\subsection{Shallow Copy \& Deep Copy}
mutable value는 기본적으로 값이 복사가 되지 않는다. 아래와 같은 예제를 보자.

\begin{minipage}{\textwidth}
\begin{tcolorbox}[colframe=black, colback=white]
\begin{minted}{python}
a = [1,2,3]
b = a
b.append(4)
print(a)
\end{minted}
\end{tcolorbox}
\textbf{결과: }
\fbox{\texttt{[1,2,3,4]}}
\end{minipage}

위의 예제는 a라고 하는 list를 b로 복사한 뒤, b의 뒷부분에 4를 추가하고 있다. 이때, a를 출력해보면 4가 추가되어 있는 것을 알 수 있다. 즉, b에 복사가 원활하게 이루어지지 않은 것이다.

다행히도, copy라고 하는 module에는 전체 값을 복사하게 해주는 함수 deepcopy()가 있다. 다음은 이 함수를 사용하여 mutable의 일종인 list를 복사하는 예제이다.

\begin{tcolorbox}[colframe=black, colback=white]
\begin{minted}{python}
import copy
a = [1,2,3]
b = copy.deepcopy(a)
\end{minted}
\end{tcolorbox}

이제, b는 a와는 독립적인 새로운 list가 되었다.
