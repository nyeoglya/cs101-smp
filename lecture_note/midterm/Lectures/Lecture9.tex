\section[{9. File IO - {\it 파일 입출력}}]{9. File IO}

\subsection{File IO}

이 단원에서는 파일의 입출력에 대해 다룬다. 참고적으로, IO는 Input and Output의 줄임말이다.

앞서 다루었던 input()과 print()는 키보드 입력 및 모니터 출력을 다루는 함수였다. 파일 입출력은 비슷한 일을 파일을 통해 해결한다. 입력을 키보드로 수행하는 것이 아닌, 파일 내부의 값을 통해 수행하고, 출력도 마찬가지로 파일에 쓰기 연산을 통해 모니터에 출력하지 않고 수행한다.

\textbf{콘솔}\textit{\textsuperscript{Console}}은 입출력을 위한 물리적인 장치로 키보드, 모니터 등이 여기에 포함된다. 보통 입력 받기 위한 인터페이스가 존재하게 되며 이러한 인터페이스 중에서도 CLI를 \textbf{터미널}\textit{\textsuperscript{Terminal}}이라고 부른다.

기본적으로 open 함수를 이용해서 파일을 열 수 있다. 여는 방식(모드)에 따라 파일을 읽거나 쓰는 등 다양한 동작을 수행할 수 있다. 아래는 파일을 읽는 예시이다.

\begin{minipage}{\textwidth}
\begin{tcolorbox}[colframe=black, colback=white]
\begin{minted}{python}
f = open('test.txt', 'r')
print(f)
\end{minted}
\end{tcolorbox}
\textbf{결과: }
\fbox{\texttt{10}}
\end{minipage}

아래는 가능한 읽기 형식 모드를 나타낸 것이다.

\begin{longtable}[]{@{}p{4cm} p{10cm}@{}}
    \toprule
    읽기 형식 & 설명\\
    \midrule
    \endhead
    r (reading) & 읽기, 기본값\\
    w (writing) & 덮어쓰기, 파일 없으면 생성\\
    a (append) & 새로운 텍스트를 추가\\
    b (binary) & 바이너리를 읽고 씀\\
    \bottomrule
\end{longtable}

아래는 파일 입출력 관련 함수를 나타낸 것이다.

\begin{longtable}[]{@{}p{4cm} p{10cm}@{}}
    \toprule
    함수 이름 & 설명 \\
    \midrule
    \endhead
    read() & 모든 파일의 데이터를 한 번에 읽음. \\
    readline() & 파일을 한 줄씩 읽음. \\
    readlines() & 파일을 전부 읽어서 리스트를 반환함. 이때, 파일이 줄바꿈으로 구분되어 있었다면 리스트에 \textbackslash n도 추가되어 있음. \\
    write(x) & 파일에 문자열 데이터를 씀. \\
    writelines(list) & 파일에 문자열 리스트를 한번에 씀. 자동 줄 바꿈이 없기 때문에, 문자열 리스트 각 원소의 끝 부분마다 수동으로 \textbackslash n 넣어주어야 줄 바꿈이 수행됨. \\
    close() & 파일을 닫음. \\
    \bottomrule
\end{longtable}

항상 프로그램이 종료되기 전에 close() 함수를 이용하여 파일을 닫아주어야 한다. 그렇지 않으면 파일이 손상되는 등의 문제가 발생할 수도 있다.

\begin{itemize}
    \item with 구문을 이용하면 자동으로 열고 닫는 작업을 수행할 수 있다. 아래는 그 예시이다.
\begin{tcolorbox}[colframe=black, colback=white]
\begin{minted}{python}
with open('test.txt', 'r') as file:
    print(file.read())
\end{minted}
\end{tcolorbox}
    이 코드는 test.txt 파일을 r 모드로 열어서 읽고, 그것을 화면에 출력한 다음 파일을 닫는 동작을 수행한다.
\end{itemize}
