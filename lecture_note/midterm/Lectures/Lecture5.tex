\section[{5. Function and Module - {\it 함수와 모듈}}]{5. Function and Module}

\subsection{Function}

\textbf{함수}\textit{\textsuperscript{function}}는 독립적인 기능을 수행하는 프로그램의 단위이다. 즉, 특정 작업을 수행하는 명령어의 모음에 이름을 붙이는 것이다. 함수는 매개변수를 받고, 반환 값을 돌려줄 수 있다. 여러 값 반환할 경우에, 값이 tuple 자료형으로 반환된다.

\begin{itemize}
    \item 장점: 읽기 쉽고 이해하기 쉬우며, 코드 중복 막음.
\end{itemize}
함수를 선언하게 되면 메모리에 함수 객체가 생성되며 그 후, 이를 가리키는 참조가 생긴다.\\
list같은 mutable 객체는 매개변수로 함수에 넘겨주면 원래 리스트에 영향을 준다. 이때, 매개변수에 새롭게 다른 값을 대입 연산하는 순간 그 매개변수는 새로운 객체로 취급되어서 더 이상 기존 변수에 영향을 주지 않는다.

아래는 매개변수 2개를 받아서 덧셈 연산을 수행하고 이를 반환하는 함수의 예제이다. 반환 값은 return 키워드를 통해 돌려줄 수 있다.

\begin{tcolorbox}[colframe=black, colback=white]
\begin{minted}{python}
def add(a, b):
    return a + b
\end{minted}
\end{tcolorbox}

\subsection{Lambda}

\textbf{익명 함수}는 말 그대로 이름이 없는 함수이다. 보통 한 줄 정도의 간단한 함수의 작성에 이용한다.

lambda 키워드를 이용하며 이름 없이 parameter와 expression만으로 구성된다. 익명 함수는 함수를 인자로 받는 함수(고차 함수) 등에 이용된다. sort가 이러한 고차 함수의 대표적인 예시로 어떠한 List의 정렬을 수행한다고 할 때, 두 원소를 비교하는 방법을 익명 함수로 제공하여 하나의 정렬 함수인 sort를 더 다양하게 쓸 수 있도록 한다. 즉, 이러한 익명 함수는 함수의 다형성을 높이는데 기여한다.

\begin{itemize}
    \item 익명 함수를 변수에 넣어서 마치 함수처럼 사용하는 예시
\begin{tcolorbox}[colframe=black, colback=white]
\begin{minted}{python}
sum_func = lambda a, b: a + b
sum_func(1,2)
\end{minted}
\end{tcolorbox}
    
    \item 정렬하는 예시
\begin{tcolorbox}[colframe=black, colback=white]
\begin{minted}{python}
sorted_list = sorted(list, key=lambda x: len(x))
\end{minted}
\end{tcolorbox}
\end{itemize}

\subsection{Variable Scope}

\textbf{스코프}\textit{\textsuperscript{scope}}는 변수에 접근 가능한 범위를 의미한다. 다음은 스코프의 종류에 따라 구분한 것이다.

\begin{itemize}
    \item \textbf{지역변수}\textit{\textsuperscript{local variable}}: 함수 내에서만 사용
    \item \textbf{전역변수}\textit{\textsuperscript{global variable}}: 프로그램 전체에서 사용
\end{itemize}

함수 내에 선언된 변수는 기본적으로 지역 변수이다. 그러나, 함수 내에 global 키워드를 사용하면, 함수 내에서도 전역 변수를 만들 수 있다. 대신 이 전역 변수를 다른 함수의 내부에서 사용하려면 똑같이 global 키워드로 이를 명시해주어야 한다. 파이썬 인터프리터는 global을 붙이지 않으면 local 변수인지 체크하고, 그 후에 해당하는 지역 변수가 없으면 오류를 반환한다.

아래는 함수 내에 global 키워드를 붙이고 그것을 다른 함수에서 사용하는 예시이다.

\begin{minipage}{\textwidth}
\begin{tcolorbox}[colframe=black, colback=white]
\begin{minted}{python}
def fun1():
    global x
    x = 10

def fun2():
    global x
    print(x)

fun1()
fun2()
\end{minted}
\end{tcolorbox}
\textbf{결과: }
\fbox{\texttt{10}}
\end{minipage}

위의 예시에서 볼 수 있는 것처럼, 함수 내에서 global 키워드를 사용해야 변수 x의 값을 공유하는 것을 확인할 수 있다.

\subsection{Module}

\textbf{모듈}\textit{\textsuperscript{module}}은 함수나 변수, 클래스 등을 모아 놓은 하나의 파일이다. 모듈 내부에는 파이썬 문법에 맞는 명령어들이 저장된다. 다음은 이미 정의되어 있는 여러 모듈의 예시이다.

\begin{itemize}
    \item keyword module: python의 예약어가 저장되어 있는 모듈.
    \begin{longtable}[]{@{}p{4cm} p{10cm}@{}}
    \toprule
        함수 이름 & 설명 \\
        \midrule
        \endhead
        iskeyword(x) & x가 예약어인지 판별함 \\
        kwlist & 모든 예약어 리스트 \\
    \bottomrule
    \end{longtable}

    \item random module: 무작위 값들과 관련한 함수가 저장되어 있는 모듈.
    \begin{longtable}[]{@{}p{4cm} p{10cm}@{}}
    \toprule
        함수 이름 & 설명 \\
        \midrule
        \endhead
        randint(a,b) & a부터 b까지 무작위 수 \\
        random() & 0부터 1까지 무작위 수 \\
        shuffle(x) & list x를 섞음 \\
        choice(x) & list x에서 원소 하나를 선택 \\
    \bottomrule
    \end{longtable}
\end{itemize}
