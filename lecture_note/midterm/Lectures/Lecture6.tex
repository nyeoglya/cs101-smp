\section[{6. Operator and Expression - {\it 연산자와 표현식}}]{6. Operator and Expression}

\subsection{Operator}

\textbf{연산자}\textit{\textsuperscript{Operator}}는 Python에서 연산을 수행할 수 있도록 정의된 기호나 키워드를 의미한다. 연산자는 받는 값의 개수에 따라 단항 연산자, 이항 연산자로 구분할 수 있다. 파이썬에서 연산자의 종류는 아래와 같다.

\begin{itemize}
    \item \textbf{산술}\textit{\textsuperscript{Arithmetic}}: 기본적으로 수의 연산을 다루는 연산자.
    \item \textbf{관계}\textit{\textsuperscript{Relational}}: 양 변의 관계를 다루는 연산자. 조건에 부합하면 true, 부합하지 않으면 false를 반환한다.
    \item \textbf{논리}\textit{\textsuperscript{Logical}}: 논리식을 다룬다. 논리식이 참이면 true, 참이 아니면 false를 반환한다.
    \item \textbf{대입}\textit{\textsuperscript{Assignment}}: 연산자의 오른쪽 항을 먼저 계산 후에, 그 결과를 왼쪽에 넣는 연산자.
    \item \textbf{단축 대입}: 산술 연산자와 대입 연산자를 합쳐놓은 연산자.
\end{itemize}

\begin{longtable}[]{@{}p{3cm} p{11cm}@{}}
    \toprule
        연산자 형태 & 종류 \\
        \midrule
        \endhead
        산술 & +, -, *(곱하기), /(나누기), **(거듭제곱)\newline
        //(정수 나누기 몫), \%(정수 나누기 나머지) \\
        관계 & ==(양 변이 같은지 판별), !=(양 변이 다른지 판별)\newline
        \textless(우변이 큰지 판별), \textless=(우변이 크거나 같은지 판별)\newline
        \textgreater(좌변이 큰지 판별), \textgreater=(좌변이 크거나 같은지 판별) \\
        논리 & and(양 변이 둘 다 참인지 판별)\newline
        or(양 변 중 하나라도 참인지 판별)\newline
        not(거짓인지 판별) \\
        대입 & = \\
        단축 대입 & +=, -=, *=, /=, //=, \%=, **= \\
    \bottomrule
\end{longtable}

논리 연산자에서 not은 단항 연산자임을 유의해야 한다.

관계 연산자는 아래처럼 여러 연산자를 동시에 중복하여 쓸 수 있다. 각각의 경우에 중복된 하나의 식이 어떻게 해석되는지는 아래와 같다.

\begin{itemize}
    \item 'a < b < c'는 'a < b' and 'b < c'로 해석된다.
    \item '1 > 0 == 0'는 '1 > 0' and '0 == 0'으로 해석된다.
\end{itemize}

\subsection{Expression}

\textbf{표현식}\textit{\textsuperscript{Expression}}은 값을 산출하는 하나의 코드 조각의 단위이다. 표현식은 하나 이상의 값으로 표현된다. 즉, 표현식은 평가할 수 있는 구문이다.

일상생활에서 이용하는 사칙연산 구문이 대표적인 예시이다. '1+2+3'과 같은 것은 결과적으로 6이라고 하는 하나의 값으로 표현이 가능하다. 프로그래밍에서는 함수 호출, 변수가 포함된 식, 배열 할당 연산 등도 모두 표현식에 포함된다.

중요한 점은, 표현식은 결과적으로 하나의 '값'으로 줄어든다는 것이다. 함수 호출의 경우, 함수의 반환값으로 줄어들며, 

반면에, \textbf{구문}\textit{\textsuperscript{Statement}}은 '실행가능한' 프로그램의 단위로, 특정 작업을 수행하는 독립적인 코드 구문이다. 프로그래밍을 하면서 인터프리터가 이해하고 실행할 수 있는 모든 문자열은 다 구문이다. 문법적으로는 파이썬의 코드 한 줄이나 한 블록이 전부 구문이라고 할 수 있다.

할당, 조건, 반복, 함수 정의, 반환, 모듈 import가 구문에 해당되며, 구문은 흔히 한 개 이상의 표현식을 포함하는 경우가 많다. 즉, 구문은 표현식을 포함하는 더 큰 개념이다.
