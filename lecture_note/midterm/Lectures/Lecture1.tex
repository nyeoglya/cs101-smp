% TeX root = ../Main.tex

% First argument to \section is the title that will go in the table of contents. Second argument is the title that will be printed on the page.
\section[1. Fundamental Words - {\it 꼭 알아야 하는 용어들}]{1. Fundamental Words}

\begin{itemize}

\item Array: 한 가지 type의 값만을 넣을 수 있다.
\item List: 다른 여러 type의 값들을 넣을 수 있다.
\item ASCII: American Standard Code for Information Interchange, 7bit.
\item Module: 큰 문제를 기능별 작은 단위로 나눈 것. 독립적으로 수행할 수 있는 프로그램 단위이다.

\item Scratch, App Inventor, Entry: MIT, 네이버가 제작한 블록 코딩 플랫폼이다.
\item C, C++: OS(Operating System)를 만들기 위해 개발된 언어이다.
\item Python, JAVA: 보통 네트워크 상황에서 자주 쓰이며, 분산된 개인 PC를 연결하려는 목적으로 개발되었다. OOP \& procedural 언어이다.

\item DOS(Disk OS): disk에 읽기, 쓰기 등의 명령을 수행하는 프로그램. TUI(Terminal User Interface) 기반이다.
\item UNIX: multi user OS. C 언어로 작성되어 이식성이 높다는 특징이 있다. 보통 서버에 자주 이용된다. 기업 및 대규모 기관에서 주로 사용되며, 구현 코드는 비공개.
\begin{itemize}
  \item UNIX 철학: 단일 목적의 소형 프로그램이 협력하여 작업 수행하는 것이 목표.
\end{itemize}
\item Linux: Unix와 유사한 오픈소스 무료 프로젝트이다. 다양한 배포판이 존재함.
\item MacOS: 애플이 개발한 OS. 사용자 친화적이고 보안이 강력하며, UNIX 기반이다.
\item Windows: MS가 개발한 OS. 넓은 호환성을 갖는다.

\item Git: 분산 버전 관리 시스템. 여러 사람이 공동으로 연구할 때 이메일로만 주고받을 수 없으니, 분산 버전 관리 시스템에 저장하여 사용한다.
\item GitHub: MS에서 인수한 후에 관리 중인, GIT 이용하는 소프트웨어 개발 프로젝트를 위한 하나의 플랫폼. 코드 저장소를 호스팅하고 버전 관리를 간편하게 할 수 있게 도와준다.
\begin{itemize}
  \item repository: 코드 파일, 수정 이력, 관련 설정 등을 저장하는 하나의 저장소이다.
  \item branch: 코드의 변경 사항을 메인 프로젝트에 영향을 주지 않고 실험할 수 있게 해주는 일종의 독립된 작업 공간. 하나의 repository 안에 여러 branch가 있으며, 기본적으로 프로젝트는 main branch에서 관리된다.
  \item commit: 저장소의 파일 변경 사항을 기록한다.
  \item pull request: 특정 branch를 main branch 병합해달라는 요청이다.
  \item issue: 프로젝트의 버그 제보, 기능이나 개선 사항을 제안하는 데 사용한다.
\end{itemize}

\item Tensorflow, Keras, CNTK, PyTorch: 인공지능의 기초, ML을 지원하는 오픈소스 라이브러리이다.
\begin{itemize}
  \item Tensorflow: tensor라고 하는 다차원 배열을 데이터 구조로 하는 라이브러리이다. 구글에서 관리 중.
  \item Keras: Tensorflow에 통합된 고수준 API. 딥러닝 모델을 쉽고 빠르게 만들게 해준다.
  \item CNTK: MS에서 만든 라이브러리. 분산 훈련을 지원하여 GPU 활용한 대규모 학습이 쉽게 가능하다.
  \item PyTorch: Facebook에서 만든 라이브러리. pythontic한 코드 스타일, 자동 미분과 같은 기능으로 신경망 훈련을 크게 단순화시킨다.
\end{itemize}
\item AlphaGo: Google DeepMind가 개발한 인공지능 프로그램으로 바둑을 두기 위해 설계되었다. 기계 학습과 신경망을 활용하여 사람보다 뛰어난 평가를 받았으며, 강화학습으로 학습되었다.

\item OpenAI: 2015년에 설립된 AI 연구소. AI의 잠재력을 최대한 활용하여 인류에게 이익을 제공하는 것이 목적. 책임감 있고 안전한 AI 개발에 초점을 둔다. 샘 올트만, 일론 머스크 등이 설립하였으며, 일반인공지능(AGI) 존재 위험에 대한 염려가 설립 동기이다.
\item transformer 아키텍처: 기존의 RNN, LSTM, CNN과는 달리 attention 매커니즘을 사용하여 sequence를 처리한다. self-attention은 입력 sequence의 각 요소 간의 상대적인 중요성을 계산하여 문맥을 보존하고, 긴 거리의 종속성을 캡처하는데 효과적이다.
\item GPT3(Generative Pretrained Transformer): OpenAI에서 제작한 모델이며, transformer 아키텍처를 기반으로 하고 있다.
\item SORA: OpenAI에서 제작한 영상 제작 모델. 정적인 노이즈에서 시작하여 여러 단계를 거쳐 노이즈를 제거하여 점점 비디오 변형시키는 diffusion 모델이며, GPT처럼 transformer 아키텍처를 사용하여 뛰어난 확장 성능을 제공한다.

\item 디지털 트윈: 컴퓨터에 물리적인 물체를 정확하게 반영하도록 설계된 가상의 모델을 만들고, 현실에서 발생할 수 있는 상황을 컴퓨터로 시뮬레이션하여 결과를 미리 예측하는 기술.
\item 메타버스: 가상 혹은 디지털 세계를 의미하는 용어. 현실 세계와 유사한 공간을 생성하고 사용자들이 가상으로 상호작용하는 환경을 지칭한다.
\begin{itemize}
    \item 디지털 트윈, 메타버스의 차이점: 디지털 트윈은 보통 현실을 모델링하여 분석하고 개선하는데 그 목적을 두지만, 메타버스는 가상 시스템에서 상호작용하는데 그 목적을 둔다.
\end{itemize}

\item NFT: 블록체인 기술을 이용하여 서로 대체할 수 없는 고유한 특성을 가진 디지털 자산이다.
\item 블록체인: 분산 데이터베이스 기술. 연결된 데이터를 블록이라는 단위로 묶어 체인 형태로 관리한다. 각 블록이 이전 블록의 고유한 식별자 포함하고 있고 데이터의 변경이나 위조를 방지한다.
\begin{itemize}
  \item 탈중앙화: 데이터를 체인 형태로 관리하여 중앙 서버에 집중되는 것을 막는다.
  \item 무결성: 데이터 변경시 이전 블록부터 현재까지 모든 블록을 변경해야 하기에 변경이 사실상 불가능하다.
  \item 투명성: 모든 거래 기록을 공개하여 신뢰성 높인다.
\end{itemize}

\item AWS, Google Cloud, Azure: 클라우드 컴퓨팅 서비스를 제공하는 주요 플랫폼.
  \begin{itemize}
    \item AWS: 가장 큰 플랫폼이며 웹 호스팅 등에 주로 이용된다.
    \item Google Cloud: 구글에서 제공하는 머신 러닝 및 인공지능에 강점을 두는 플랫폼이다.
    \item Azure: MS에서 만들었고 기업용 솔루션과의 통합이 강점이다.
  \end{itemize}
\end{itemize}
